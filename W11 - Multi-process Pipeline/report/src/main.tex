%! Author = fraser
%! Date = 03/04/2021

% Preamble
\documentclass{article}

% Packages
\usepackage{fancyvrb}
\usepackage{pdflscape}
\usepackage{graphicx}
\usepackage{float}
\usepackage[utf8]{inputenc}
\usepackage{xparse}
\usepackage{caption}
\usepackage{subcaption}
\usepackage{geometry}
\usepackage{minted}
\usepackage{amssymb}
\usepackage{tikz}

\usetikzlibrary{arrows,shapes.gates.logic.US,shapes.gates.logic.IEC,calc}
\newcommand{\lnor}{\bar{\hspace*{0.3em}\lor\hspace*{0.2em}}}
\newcommand{\lnand}{\bar{\hspace*{0.3em}\land\hspace*{0.2em}}}
\newcommand{\lxor}{\oplus}
\newcommand{\lxnor}{\hspace*{0.3em}\underbar{\hspace*{-0.3em}\lnor\hspace*{-0.2em}}\hspace*{0.2em}}
\tikzstyle{branch}=[fill,shape=circle,minimum size=3pt,inner sep=0pt]
\newcommand{\size}[2]{{\fontsize{#1}{0}\selectfont#2}}
\newenvironment{sizepar}[2]
{\par\fontsize{#1}{#2}\selectfont}
{\par}

% Document
\begin{document}

    \nocite{*}

    \begin{center}
        \Huge
        CS2002 Week 11 Practical

        \vspace{0.5cm}

        \textbf{Logic}

        \vspace{1cm}
        \LARGE
        19th April 2021

        \large
        \vspace{1.5cm}

        \textbf{Matriculation Number: 200002548}

        \vspace{0.5cm}

        \textbf{Tutor: Marco Caminati}

    \end{center}

    \vspace*{3cm}

    \tableofcontents

    \newpage
    \section{Introduction}
    The aim of this practical was to gain experience with processes, multi-process pipelines and simple inter-process communication (IPC)
    in C. In this practical we had to design and implement a data processing pipeline consisting of a number of processing stages, connected in series
    and use pipes for inter-process communication.
    The pipeline implementation should allow for an arbitrary number of stages to be added to the pipeline, with each stage getting its own dedicated process.

    \section{Design and Implementation}
    \subsection{Pipeline Struct}
    The \verb+Pipeline+ struct is defined within \verb+pipeline.h+.
    The struct contains pointers called \verb+stages+ and \verb+pipes+ that point to dynamically allocated arrays.
    The \verb+stages+ array contains pointers that point to functions to be included in the pipeline and given an independent process.
    The \verb+pipes+ array contains further arrays of length two that contain the file descriptors associated with the pipes.
    These arrays of length two use \verb+typedef+ so that they can simply be refered to as a \verb+Pipe+ to increase readability.

    \begin{thebibliography}{10}
        \bibitem{numeric}
        WillNess, \textit{Sieve of Eratosthenes}, 14-03-2013, accessed 19-04-2021, \\\texttt{https://en.wikipedia.org/w/index.php?title=Sieve_of_Eratosthenes&oldid=544116469}

        \bibitem{numeric}
        WhozCraig, \textit{Using pipe to pass integer values between parent and child}, 12-10-2012, accessed 19-04-2021, \\\texttt{https://stackoverflow.com/a/12864595}

        \bibitem{numeric}
        davmac, \textit{What exactly happens when you create a pipe after a fork() command?}, 27-03-2018, accessed 19-04-2021, \\\texttt{https://stackoverflow.com/a/49520618}

        \bibitem{numeric}
        The Linux Documentation Project, \textit{Creating Pipes in C}, 29-03-1996, accessed 19-04-2021, \\\texttt{https://tldp.org/LDP/lpg/node11.html}

    \end{thebibliography}
\end{document}